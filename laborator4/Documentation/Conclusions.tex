\section*{Concluzie}
\phantomsection

\par Efectuând această lucrare de laborator am învățat multe lucruri noi în toate domeniile acoperite, deoarece pân acum nu am avut experiență în crearea aplicațiilor mobile și nici Java. 
\par Din cunoștințele acumulate pe parcursul elaborării aplicației fac parte:
\begin{itemize}
 \item Ciclul de viață a aplicațiilor Android;
 \item Cum se configurează un fișier \textit{Manifest};
 \item Ce sunt \textit{Activity} și \textit{Fragment};
 \item Ce reprezintă un \textit{Adapter};
 \item Cunoștințe de bază în Java;
 \item Ce trebuie să conțină fișierul \textit{.gitignore} pentru proiectele create în Android Studio.
\end{itemize}
\par De m-am familiarizat cu mediul de dezvoltare Android Studio și m-am folosit de instrumentul său puternic de lucru cu sistemul de control al versiunilor \textit{git}. El permite adăugarea unor commituri noi sau schimbarea brach-ului curent în doar câtevai click-uri. Eu l-am găsit a fi foarte comod în folosință.

\clearpage