\section*{Concluzie}
\phantomsection

În această lucrare am utilizat sistemul de control al versiunilor GIT. Acest instrument lucrează în mod implicit cu modulul de interacțiune CLI. Astfel toate acțiunile 
au fost efectuate folosind comenzile prezente în git. Unele din aceste comenzi sunt:
\begin{enumerate}[leftmargin=4cm]
\item [\textbf{git init}] Inițializarea unui repozitoriu
\item [\textbf{git config}] Configurarea unui repozitoriu
\item [\textbf{git checkout -b}] Crearea unui branch
\item [\textbf{git brach}] Afișarea listei de branch-uri
\item [\textbf{git add}] Înregistrează schimbarile
\item [\textbf{git commit}] Crează o versiune nouă cu schimbarile înregistrate
\item [\textbf{git remote add}] Adaugă remote
\item [\textbf{git push}] Salvează schimbarile local pe repozitoriu extern
\item [\textbf{git merge}] Execută merge între 2 branch-uri
\item [\textbf{git tag}] Modificarea tag-urilor din acest repozitoriu
\end{enumerate}
Un element la fel de important în crearea și întreținerea repozitoriului GIT este fișierul \textit{.gitignore}. Acesta permite excluderea altor fișiere care nu ar trebui să fie împărțite cu alți dezvoltatori care lucrează la același repozitoriu.

\clearpage